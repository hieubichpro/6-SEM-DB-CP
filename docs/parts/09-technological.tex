\section{Технологический раздел}
В данной части рассматривается выбор средств реализации, описывается структура классов программы и приводится интерфейс программного обеспечения.

\subsection{Средства реализации}

Для написания данной курсовой работы был выбран язык C\#~\cite{cpp-lang}.
Выбор данного языка программирования обусловлен следующим образом:
\begin{itemize}
	\item C\# --- объектно-ориентированный язык, а именно такая методология
	программирования была выбрана для разработки программы;
	\item в данном языке имеется большое количество библиотек и шаблонов,
	позволяющих не тратить время на изобретение готовых конструкций;
	\item C\# имеет встроенный механизм LINQ, предоставляющий возможности выполнения запросов к базе данных на уровне языка.
\end{itemize}

В качестве среды разработки был использован Visual Studio 2022~\cite{qt-creator}.
Данный выбор обусловлен следующими факторами:
\begin{itemize}
	\item в Visual Studio есть возможность быстрого создания интерфейса с помощью WinForms;
	\item предоставляет шаблоны, которые облегчают процесс написания и отладки проекта.
\end{itemize}

\subsection{Выбор СУБД}
В качестве СУБД был выбран PostgreSQL~\cite{cpp-lang}.
Выбор данной СУБД обусловлен следующим образом:

\begin{itemize}
	\item PostgreSQL обладает обширным набором функций, включая поддержку различных типов данных (встроенных, пользовательских), оконных функций и многого другого;
	\item PostgreSQL известен своей высокой производительностью и надежностью. Он обеспечивает эффективную обработку запросов, поддержку параллельной обработки и оптимизацию запросов;
	\item PostgreSQL обладает возможностью масштабирования как вертикально, так и горизонтально;
	\item PostgreSQL обеспечивает поддержку транзакций с соблюдением принципов ACID (атомарность, согласованность, изолированность, долговечность), что обеспечивает целостность данных и надежность операций.
\end{itemize}

\subsection{Создание базы данных}

\subsection{Описание интерфейсов}

\subsection*{Вывод}
Было приведено описание структура программы, выбраны средства реализации программного обеспечения, приведены листинги кода, продемонстрирован интерфейс программы и представлены результаты работы программы. 
