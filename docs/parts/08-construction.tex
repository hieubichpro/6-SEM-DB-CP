\section{Конструкторский раздел}

\subsection{Проектирование базы данных}

На основе выделенных ранее сущностей спроектированы следующие объекты базы данных:
\begin{enumerate}
	\item users --- таблица пользователей;
	\item countries --- таблица стран;
	\item stadiums --- таблица стадионов;
	\item feedbacks --- таблица отзывов гостей;
	\item requests --- таблица заявок пользователей;
	\item leagues --- таблица турниров;
	\item clubs --- таблица клубов;
	\item matches --- таблица матча;
\end{enumerate}

На основе диаграммы сущность-связей, представленной на рисунке \ref{img:ERdiagram} определяются стуктуры столбцов, их типы и так же органичения.

\begin{table}[H]
	\begin{center}
		\caption{Информация о столбцах таблицы пользователей}
		\begin{tabular}{|c|c|c|c|}
			\hline
			Столбец & Тип данных & Ограничения & Значение \\
			\hline
			id & serial & PRIMARY KEY & Идентификатор \\
			\hline
			lastname & VARCHAR(32) & NOT NULL & Фамилия \\
			\hline
			firstname & VARCHAR(32) & NOT NULL & Имя\\
			\hline
			login & VARCHAR(64) & NOT NULL &  Логин \\
			\hline
			password & VARCHAR(64) & NOT NULL & Пароль \\
			\hline
			role & VARCHAR(64) & NOT NULL & Права доступа \\
			\hline
		\end{tabular}
		\label{table:db:users}
	\end{center}
\end{table}

\begin{table}[H]
	\begin{center}
		\caption{Информация о столбцах таблицы турнира}
		\begin{tabular}{|c|c|c|c|}
			\hline
			Столбец & Тип данных & Ограничения & Значение \\
			\hline
			id & serial & PRIMARY KEY & Идентификатор \\
			\hline
			name & VARCHAR(32) & NOT NULL & Название турнира \\
			\hline
			rating & DOUBLE & NOT NULL & Рейтинг турнира\\
			\hline
			iduser & INT & NOT NULL &  ID пользователя \\
			\hline
			idcountry & INT & NOT NULL & ID страны \\
			\hline
		\end{tabular}
		\label{table:db:league}
	\end{center}
\end{table}

\begin{table}[H]
	\begin{center}
		\caption{Информация о столбцах таблицы команды}
		\begin{tabular}{|c|c|c|c|}
			\hline
			Столбец & Тип данных & Ограничения & Значение \\
			\hline
			id & serial & PRIMARY KEY & Идентификатор \\
			\hline
			name & VARCHAR(32) & NOT NULL & Название команды \\
			\hline
			idcountry & INT & NOT NULL & ID страны \\
			\hline
		\end{tabular}
		\label{table:db:club}
	\end{center}
\end{table}

\begin{table}[H]
	\begin{center}
		\caption{Информация о столбцах таблицы стадиона}
		\begin{tabular}{|c|c|c|c|}
			\hline
			Столбец & Тип данных & Ограничения & Значение \\
			\hline
			id & serial & PRIMARY KEY & Идентификатор \\
			\hline
			name & VARCHAR(32) & NOT NULL & Название стадиона \\
			\hline
			capacity & INT & NOT NULL & Количество мест \\
			\hline
			idcountry & INT & NOT NULL & ID страны \\
			\hline
		\end{tabular}
		\label{table:db:stadium}
	\end{center}
\end{table}

\begin{table}[H]
	\begin{center}
		\caption{Информация о столбцах таблицы отзыва}
		\begin{tabular}{|c|c|c|c|}
			\hline
			Столбец & Тип данных & Ограничения & Значение \\
			\hline
			id & serial & PRIMARY KEY & Идентификатор \\
			\hline
			commment & VARCHAR(256) & NOT NULL & Коментарий \\
			\hline
			mark & INT & NOT NULL & Оценка \\
			\hline
			iduser & INT & NOT NULL & ID пользователя \\
			\hline
			idleague & INT & NOT NULL & ID турнира \\
			\hline
		\end{tabular}
		\label{table:db:feedback}
	\end{center}
\end{table}

\begin{table}[H]
	\begin{center}
		\caption{Информация о столбцах таблицы заявки}
		\begin{tabular}{|c|c|c|c|}
			\hline
			Столбец & Тип данных & Ограничения & Значение \\
			\hline
			id & serial & PRIMARY KEY & Идентификатор \\
			\hline
			timecreated & DATE & NOT NULL & Время создания \\
			\hline
			iduser & INT & NOT NULL & ID пользователя \\
			\hline
			idleague & INT & NOT NULL & ID турнира \\
			\hline
			idclub & INT & NOT NULL & ID клуба \\
			\hline
		\end{tabular}
		\label{table:db:request}
	\end{center}
\end{table}

\begin{table}[H]
	\begin{center}
		\caption{Информация о столбцах таблицы матча}
		\begin{tabular}{|c|c|c|c|}
			\hline
			Столбец & Тип данных & Ограничения & Значение \\
			\hline
			id & serial & PRIMARY KEY & Идентификатор \\
			\hline
			goalhome & DATE & NOT NULL & Голы домашней \\
			\hline
			goalguest & INT & NOT NULL & Голы гостевой \\
			\hline
			idleague & INT & NOT NULL & ID турнира \\
			\hline
			idhome & INT & NOT NULL & ID домащ клуба \\
			\hline
			idguest & INT & NOT NULL & ID гост клуба \\
			\hline
		\end{tabular}
		\label{table:db:match}
	\end{center}
\end{table}

\subsection*{Вывод}
Были описаны требования к программному обеспечению, алгоритмы для построения сцены в пространстве изображения, изменения положения объекта в пространстве, построение камеры и ее проекций. Также описаны процедурные текстуры, проективные текстуры и алгоритм моделирования неровностей.